\subsubsection{Principles}
The \textbf{Circle} is perhaps the most well known Conic Section and by far the simplest. Because of this simplicity, satellites often are put in as circular orbits as possible.
\begin{center}
\begin{tikzpicture}
    \newcommand{\da}{5}
    \newcommand{\db}{5}
    \newcommand{\dc}{0}
    \newcommand{\dl}{5}
    %\draw (0,0) ellipse (\da cm and \db cm);
    \begin{scope}[scale=.5]
    \draw [domain= 0:360, scale=1,samples=500,shift={(\dc,0)}] plot (\x:{\dl / ((\dc / \da) * cos(\x) + 1)});
    \draw[teal] (0,0) -- (0.7071 * \da, -0.7071 * \da);
    \draw[blue] (0,0) -- (- \da,0);
    \draw[red] (0,0) -- (0,\db);
    \draw[orange] (\dc,0) -- (\da,0);
    \draw[purple, thick] (\dc,0) -- (\dc,\dl);
    \filldraw [black] (\dc,0) circle (1pt) node[anchor = north east] {focus};
    \node[anchor = north east] at (\da * 0.7071/2, \da * -0.7071/2) {\textcolor{teal}{$r_1$}};
    \node[anchor = east] at (\dc, \dl / 2) {\textcolor{purple}{$l$}};
    \node[anchor = north] at (\dc/ 2 +\da / 2, 0) {\textcolor{orange}{$p$}};
    \node[anchor = north] at (- \da / 2, 0) {\textcolor{blue}{$a$}};
    \node[anchor = west] at (0, \db/2) {\textcolor{red}{$b$}};
    \end{scope}
\end{tikzpicture}
\end{center}

Defining Property: All points are equidistant from the center.

\begin{multicols}{2}
\noindent
 $$r = \text{\textit{constant}}$$
 $$r = \textcolor{teal}{r_1} = \textcolor{brown}{r_2}$$
 $$r = \textcolor{blue}{a} = \textcolor{red}{b}$$
 $$x^2 + y^2 = \textcolor{black}{r}^2$$
 $$x^2 + y^2 = \textcolor{teal}{r_1}^2$$
 $$\left(\frac{x}{\textcolor{teal}{r_1}}\right)^2 + 
 \left(\frac{y}{\textcolor{teal}{r_1}}\right)^2 = 1$$
 $$\left(\frac{x}{\textcolor{blue}{a}}\right)^2 + \left(\frac{y}{\textcolor{red}{b}}\right)^2 = 1$$
 \end{multicols}
 
 Here we can see that a circle is actually just a special case of an ellipse, where Semi Major Axis equals Semi Minor Axis. We can take advantage of this fact to find/verify the domain of the shape.
 $$\textcolor{blue}{a} = \textcolor{red}{b}$$
 $$\textcolor{red}{b} = \textcolor{blue}{a}\sqrt{1-\textcolor{black}{e}^2}$$
 $$\textcolor{blue}{a} = \textcolor{blue}{a}\sqrt{1-\textcolor{black}{e}^2}$$
 $$1^2 = 1-\textcolor{black}{e}^2$$
 $$\textcolor{black}{e} = 0$$
 
 Not only is a circle just an ellipse with equivalent Major and Minor Axis, we can prove that it is an ellipse with an eccentricity of zero.
 
 Additionally, since all points are equidistant to the focus, the Semi Latus Rectum is also equal to radius.
