\chapter{Introduction and Axioms}
\section{Intro to Orbits}

\section{Identities}

    Identities of an orbit:\\\\
    
    The construction of an orbit (Orbital  Trajectory) can be split into two parts: Geometry and Physics. Each of these takes some properties to determine the characteristics of the orbit. These properties are called ``Identities", since combined they can identify a unique orbit.
    
    The geometric identity of the orbit can be defined with almost any two properties of the geometry. The two we will be using are Eccentricity (used for defining the shape of the orbit) and Semi Latus Rectum (used for defining the scale of the orbit) These properties aren't traditionally used for orbits, but become extremely useful for both geometric construction and navigation. It is also possible to substitute one of these properties for Periapsis, since it is one of the few properties that share the universal applicability with the the other two properties, and is frequently used in navigation.
    
    The physics identity is slightly easier to define, as it only requires a single property that is more standardized: the Standard Gravitational Parameter. This property allows the physics calculations to easily translate between gravitation and motion.
\\\\    
Geometric Identities:
\begin{itemize}
    \item Eccentricity: The scalar metric of how far the conic section deviates from a circle; \textcolor{black}{$e$}
    \item Semi Latus Rectum: 1/2 The length of a chord intersecting the primary focus, perpendicular to the major axis; \textcolor{purple}{$l$} 
    \item Periapsis: The shortest distance between the orbit and the body at the prime focus; \textcolor{orange}{$p$}
\end{itemize}
Physics Identities:
\begin{itemize}
    \item Standard Gravitational Parameter: The product of the mass of the body being orbited and the Universal Gravitational Constant ; \textcolor{DarkOrchid}{$\mu$}\\
\end{itemize}
Geometric Formula's used:
\begin{itemize}
    \item[] (Note: The following all have the center of the shape at the origin)
    \item Standard Circle Formula: $x^2 + y^2 = \textcolor{black}{r}^2$   $\{\textcolor{black}{e}=0\}$
    \item Standard Ellipse Formula: $\left(\frac{x}{\textcolor{blue}{a}}\right)^2 + \left(\frac{y}{\textcolor{red}{b}}\right)^2 = 1$   $\{0<\textcolor{black}{e}<1 \}$
    \item Standard Parabola Formula: $y^2=-4\textcolor{orange}{p}x$ $\{\textcolor{black}{e}=1\}$
    \item Standard Hyperbola Formula: $\left(\frac{x}{\textcolor{blue}{a}}\right)^2 - \left(\frac{y}{\textcolor{red}{b}}\right)^2 = 1$   $\{\textcolor{black}{e}>1 \}$\\
\end{itemize}
Physics Formulas used:
\begin{itemize}
    \item Conservation of Specific Mechanical Energy: $\epsilon=\epsilon_k+\epsilon_p$
    \item Conservation of Specific Angular Momentum:
    $\frac{L}{m_{satellite}}=rv_\perp$
    %\item Vis-Viva Equation: $\textcolor{ForestGreen}{v} = \sqrt{\textcolor{DarkOrchid}{\mu}(\frac{2}{\textcolor{teal}{r}} - \frac{1}{\textcolor{blue}{a}})}$
\end{itemize}

%\subsection{Parts of an Orbit}